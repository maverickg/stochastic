%&subfile
\begin{document}
\ifstandalonechapter\setcounter{chapter}{3}\fi
\chapter{Risk Neutral Measures}

Our aim in this section is to show how risk neutral measures can be used to price derivative securities.
The key advantage is that under a risk neutral measure the discounted hedging portfolio becomes a martingale.
Thus the price of any derivative security can be computed by conditioning the payoff at maturity.
We will use this to provide an elegant derivation of the Black-Scholes formula, and discuss the fundamental theorems of asset pricing.

\section{Change of measure (unfinished).}

\begin{definition}
  Let $\Omega$ be 
  Two probability measures $\P$ and $\tilde \P$ are said to be equivalent if $\P(A) = 0$ if and only if $\tilde \P(A) = 0$.
\end{definition}

The \emph{Radon-Nikodym} theorem guarantees that two measures are equivalent if and only if there exists a strictly positive random variable $Z$ TODO

\begin{itemize}
  \item
    RN theorem, $d\tilde \P = Z d\P$ notation.

  \item
    $Z > 0$ a $\P$-martingale, define $\tilde \P = \tilde P_T$ by $d\tilde P = Z(T) d\P$.

  \item
    Say $Z(T)$ is $\mathcal F_T$ measurable
\end{itemize}


\end{document}
