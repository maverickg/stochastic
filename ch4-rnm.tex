%&subfile
\begin{document}
\ifstandalonechapter\setcounter{chapter}{3}\fi
\chapter{Risk Neutral Measures}
%\ifstandalonechapter
%  \makeatletter
%   \@bsphack\begingroup
%   \def\@currentlabel{asdf}%
%   \label{t:3ipconv}%
%   \endgroup\@esphack
%  \makeatother
%\fi

Our aim in this section is to show how risk neutral measures can be used to price derivative securities.
The key advantage is that under a risk neutral measure the discounted hedging portfolio becomes a martingale.
Thus the price of any derivative security can be computed by conditioning the payoff at maturity.
We will use this to provide an elegant derivation of the Black-Scholes formula, and discuss the fundamental theorems of asset pricing.

\section{The Girsanov Theorem.}

\begin{definition}
  Two probability measures $\P$ and $\tilde \P$ are said to be equivalent if for every event $A$, $\P(A) = 0$ if and only if $\tilde \P(A) = 0$.
\end{definition}

\begin{example}
  Let $Z$ be a random variable such that $\E Z = 1$ and $Z > 0$.
  Define a new measure $\tilde\P$ by
  \begin{equation}\label{e:4pti}
    \tilde \P(A) = \E Z \one_A = \int_A Z \, d\P\,.
  \end{equation}
  for every event $A$.
  Then $\P$ and $\tilde\P$ are equivalent.
\end{example}

\begin{remark}
  The assumption $\E Z = 1$ above is required to guarantee $\tilde \P(\Omega) = 1$.
\end{remark}

\begin{definition}
  When $\tilde\P$ is defined by~\eqref{e:4pti}, we say
  \begin{equation*}
    d\, \tilde\P = Z \, d\P
    \qquad\text{or}\qquad
    Z = \frac{d\tilde\P}{d\P}\,,
  \end{equation*}
  and $Z$ is called the \emph{density} of $\tilde\P$ with respect to $\P$.
\end{definition}

\begin{theorem}[Radon-Nikodym]
  Two measures $\P$ and $\tilde\P$ are equivalent if and only if there exists a random variable $Z$ such that $\E Z = 1$, $Z > 0$ and $\tilde\P$ is given by~\eqref{e:4pti}.
\end{theorem}

The proof of this requires a fair amount of machinery from Lebesgue integration and goes beyond the scope of these notes. 
(This is exactly the result that is used to show that conditional expectations exist.)
However, when it comes to risk neutral measures, it isn't essential since in most of our applications the density will be explicitly chosen.

Suppose now $T>0$ is fixed, and $Z$ is a martingale.
Define a new measure $\tilde\P = \tilde\P_T$ by
\begin{equation*}
  d\tilde\P = d\tilde\P_T = Z(T) d\P\,.
\end{equation*}
We will denote expectations and conditional expectations with respect to the new measure by $\tilde \E$.
That is, given a random variable $X$,
\begin{equation*}
  \tilde\E X = \E Z(T) X = \int Z(T) X \, d\P\,.
\end{equation*}
Also, given a $\sigma$-algebra $\mathcal F$, $\tilde \E(X \given \mathcal F)$ is the unique $\mathcal F$-measurable random variable such that
\begin{equation}\label{e:4pav}
  \int_F \tilde \E (X \given \mathcal F) \, d\tilde\P
  = \int_F X  \, d\tilde\P\,,
\end{equation}
holds for all $\mathcal F$ measurable events $F$.
Of course, equation~\eqref{e:4pav} is equivalent to requiring
\begin{gather}\label{e:4pavPrime}
  \tag{\ref{e:4pav}$'$}
  \int_F Z(T) \tilde \E (X \given \mathcal F) \, d\P
  = \int_F Z(T) \tilde X  \, d\P\,,
\end{gather}
for all $\mathcal F$ measurable events $F$.

The main goal of this section is to prove the Girsanov theorem.
\begin{theorem}[Cameron, Martin, Girsanov]\label{t:4girsanov}
  Let $b(t) = (b_1(t), b_2(t), \dots, b_d(t))$ be a $d$-dimensional adapted process, $W$ be a $d$-dimensional Brownian motion, and define
  \begin{equation*}
    \tilde W(t) = W(t) + \int_0^t b(s) \, ds\,.
  \end{equation*}
  Let $Z$ be the process defined by
  \begin{equation*}
    Z(t) = \exp\paren[\Big]{ -\int_0^t b(s) \cdot dW(s) - \frac{1}{2}\int_0^t \abs{b(s)}^2 \, ds}\,,
  \end{equation*}
  and define a new measure $\tilde P = \tilde P_T$ by $d\tilde\P = Z(T) \, d\P$.
  If $Z$ is a martingale then $\tilde W$ is a Brownian motion under the measure $\tilde \P$ up to time $T$.
\end{theorem}
\begin{remark}
  Above
  \begin{equation*}
    \int_0^t b(s) \cdot dW(s) \defeq \sum_{i=1}^{d} b_i(s) dW_i(s)
    \qquad\text{and}\qquad
    \abs{b(s)}^2 = \sum_{i=1}^{d} b_i(s)^2\,.
  \end{equation*}
\end{remark}
\begin{remark}
  Note $Z(0) = 1$, and if $Z$ is a martingale then $\E Z(T) = 1$ ensuring $\tilde P$ is a probability measure.
  You might, however, be puzzled at need for the assumption that $Z$ is a martingale.
  Indeed, let $M(t) = \int_0^t b(s) \cdot dW(s)$, and $f(t, x) = \exp(-x - \frac{1}{2} \int_0^t b(s)^2 \, ds )$.
  Then, by It\^o's formula,
  \begin{align*}
    dZ(t)
      &= d\paren[\big]{ f(t, M(t)) }
      = \partial_t f \, dt + \partial_x f \, dM(t)  + \frac{1}{2} \partial_x^2 f \, d\qv{M}(t)
    \\
      &= -\frac{1}{2} Z(t) \abs{b(t)}^2 \, dt
	-Z(t) b(t) \cdot dW(t) 
	+ \frac{1}{2} Z(t) \abs{b(t)}^2 \, dt\,,
  \end{align*}
  and hence
  \begin{equation}\label{e:4dz}
      dZ(t) = -Z(t) b(t) \cdot dW(t)\,.
  \end{equation}
  Thus you might be tempted to say that $Z$ is always a martingale, assuming it explicitly is unnecessary.
  However, we recall from Chapter~\ref{3c:3int}, Theorem~\ref{3t:3ipconv} in that It\^o integrals are only guaranteed to be martingales under the square integrability condition
  \begin{equation}\label{e:4fin}
    \E \int_0^T \abs{Z(s) b(s)}^2 \, ds < \infty\,.
  \end{equation}
  Without this finiteness condition, It\^o integrals are only \emph{local martingales}, whose expectation need not be constant, and so $\E Z(T) = 1$ is not guaranteed.
  Indeed, there are many examples of processes $b$ where the finiteness condition~\eqref{e:4fin} does \emph{not} hold and we have $\E Z(T) < 1$ for some $T > 0$.
\end{remark}

\begin{remark}
  In general the process $Z$ above is always a super-martingale, and hence $\E Z(T) \leq 1$.
  Two conditions that guarantee $Z$ is a martingale are the Novikov and Kazamaki conditions:
  If either
  \begin{equation*}
    \E \exp\paren[\Big]{ \frac{1}{2} \int_0^t \abs{b(s)}^2 \, ds } < \infty
    \quad\text{or}\quad
    \E \exp\paren[\Big]{ \frac{1}{2} \int_0^t b(s) \cdot dW(s) } < \infty\,,
  \end{equation*}
  then $Z$ is a martingale and hence $\E Z(T) = 1$ for all $T > 0$.
  Unfortunately, in many practical situations these conditions do not apply, and you have to show $Z$ is a martingale by hand.
\end{remark}
\begin{remark}
  The components $b_1$, \dots, $b_d$ of the process $b$ are not required to be independent.
  Yet, under the new measure, the process $\tilde W$ is a Brownian motion and hence has \emph{independent} components.
\end{remark}

The main idea behind the proof of the Girsanov theorem is the following:
Clearly $\jqv{\tilde W_i, \tilde W_j} = \jqv{W_i, W_j} = \one_{i= j} t$.
Thus if we can show that $\tilde W$ is a martingale \emph{with respect to the new measure $\tilde P$}, then L\'evy's criterion will guarantee $\tilde W$ is a Brownian motion.
We now develop the tools required to check when processes are martingales under the new measure.

\begin{lemma}
  Let $0 \leq s \leq t \leq T$.
  If $X$ is a $\mathcal F_t$-measurable random variable then
  \begin{equation}\label{e:4eti}
    \tilde\E\paren[\big]{ X \given \mathcal F_s }
      = \frac{1}{Z(s)} \E \paren[\big]{ Z(t) X \given \mathcal F_s }
  \end{equation}
\end{lemma}
\begin{proof}
  Let $A \in \mathcal F_s$ and observe that
  \begin{multline*}
    \int_A \tilde\E( X \given \mathcal F_s ) \, d\tilde P
      = \int_A Z(T) \tilde\E( X \given \mathcal F_s ) \, d\P
    \\
      = \int_A \E\paren[\big]{
	  Z(T) \tilde\E( X \given \mathcal F_s ) \given \mathcal F_s} \, d\P
      = \int_A Z(s) \tilde\E( X \given \mathcal F_s )\, d\P\,.
  \end{multline*}
  Also,
  \begin{multline*}
    \int_A \tilde\E( X \given \mathcal F_s ) \, d\tilde P
      = \int_A X \, d\tilde \P
      = \int_A X Z(T) \, d\P
      = \int_A \E\paren[\big]{ X Z(T) \given \mathcal F_t } \, d\P
  \\
      = \int_A Z(t) X \, d\P
      = \int_A \E\paren[\big]{ Z(t) X \given \mathcal F_s} \, d\P
  \end{multline*}
  Thus 
  \begin{equation*}
    \int_A Z(s) \tilde\E( X \given \mathcal F_s )\, d\P
      = \int_A \E\paren[\big]{ Z(t) X \given \mathcal F_s} \, d\P\,,
  \end{equation*}
  for every $\mathcal F_s$ measurable event $A$.
  Since the integrands are both $\mathcal F_s$ measurable this forces them to be equal, giving~\eqref{e:4eti} as desired.
\end{proof}

\begin{lemma}\label{l:4pmg}
  An adapted process $M$ is a martingale under $\tilde\P$ if and only if $MZ$ is a martingale under $\P$.
\end{lemma}
\begin{proof}
  Suppose first $MZ$ is a martingale with respect to $\P$.
  Then
  \begin{equation*}
    \tilde\E( M(t) \given \mathcal F_s )
      = \frac{1}{Z(s)} \tilde\E( Z(t) M(t) \given \mathcal F_s )
      = \frac{1}{Z(s)} Z(s) M(s) = M(s)\,,
  \end{equation*}
  showing $M$ is a martingale with respect to $\P$.

  Conversely, suppose $M$ is a martingale with respect to $\tilde P$.
  Then
  \begin{equation*}
    \E\paren[\big]{ M(t) Z(t) \given \mathcal F_s }
      = Z(s) \tilde\E\paren{ M(t) \given \mathcal F_s }
      = Z(s) M(s)\,,
  \end{equation*}
  and hence $ZM$ is a martingale with respect to $\P$.
\end{proof}

\begin{proof}[Proof of Theorem~\ref{t:4girsanov}]
  Clearly $\tilde W$ is continuous and
  \begin{equation*}
    d\jqv{\tilde W_i, \tilde W_j}(t) 
      = d\jqv{W_i, W_j}(t)
      = \one_{i = j} \, dt\,.
  \end{equation*}
  Thus if we show that each $\tilde W_i$ is a martingale (under $\tilde \P$), then by L\'evy's criterion, $\tilde W$ will be a Brownian motion under $\tilde\P$.

  We now show that each $\tilde W_i$ is a martingale under $\tilde P$.
  By Lemma~\ref{l:4pmg}, $\tilde W_i$ is a martingale under $\tilde P$ if and only if $Z \tilde W_i$ is a martingale under $\P$.
  To show $Z \tilde W_i$ is a martingale under $\P$, we use the product rule and~\eqref{e:4dz} to compute
  \begin{multline*}
    d \paren[\big]{ Z \tilde W_i }
      = Z \, d\tilde W_i  + \tilde W_i \, dZ + d\jqv{Z, \tilde W_i}
  \\
      = Z \, dW_i  + Z b_i \, dt  - \tilde W_i Z b \cdot dW - b_i Z \, dt
      = Z \, dW_i - \tilde W_i Z b \cdot dW \,.
  \end{multline*}
  Thus $Z \tilde W_i$ is a martingale%
  \footnote{%
    Technically, we have to check the square integrability condition to ensure that $Z \tilde W_i$ is a martingale, and not a \emph{local martingale}.
    This, however, follows quickly from the Cauchy-Schwartz inequality and our assumption.
  }
  under $\P$, and by Lemma~\ref{l:4pmg}, $\tilde W_i$ is a martingale under~$\tilde\P$.
  This finishes the proof.
\end{proof}


\section{Risk Neutral Pricing}

Consider a stock whose price is modelled by a generalized geometric Brownian motion
\begin{equation*}
  dS(t) = \alpha(t) S(t) \, dt + \sigma(t) S(t) \, dW(t)\,,
\end{equation*}
where $\alpha(t)$, $\sigma(t)$ are the (time dependent) mean return rate and volatility respectively.
Here $\alpha$ and $\sigma$ are no longer constant, but allowed to be \emph{adapted processes}.
We will, however, assume $\sigma(t) > 0$.

Suppose an investor has access to a money market account with variable interest rate $R(t)$.
Again, the interest rate $R$ need not be constant, and is allowed to be any adapted process.
Define the \emph{discount process} $D$ by
\begin{equation*}
  D(t) = \exp\paren[\Big]{ - \int_0^t R(s) \, ds }\,,
\end{equation*}
and observe
\begin{equation*}
  dD(t) = -D(t) R(t) \, dt\,.
\end{equation*}
Since the price of one share of the money market account at time $t$ is $1/D(t)$ times the price of one share at time $0$, it is natural to consider the \emph{discounted stock price} $D S$.

\begin{definition}
  A \emph{risk neutral measure} is a measure $\tilde\P$ that is equivalent to $\P$ under which the discounted stock price process $D(t) S(t)$ is a martingale.
\end{definition}

As we will shortly see, the \emph{existence} of a risk neutral measure implies there is no arbitrage opportunity.
Moreover, \emph{uniqueness} of a risk neutral measure implies that every derivative security can be hedged.
These are the \emph{fundamental theorems of asset pricing}, which will be developed later.

Using the Girsanov theorem, we can compute the risk neutral measure explicitly.
Observe
\begin{align*}
  d\paren[\big]{D(t) S(t)}
    &= -R D S \, dt + D \, dS
    = (\alpha-R) D S \, dt + D S \sigma \, dW(t)
  \\
    &= \sigma(t) D(t) S(t) \paren[\Big]{
	  \theta(t) \, dt + dW(t) }
\end{align*}
where
\begin{equation*}
  \theta(t) \defeq \frac{\alpha(t) - R(t)}{\sigma(t)}
\end{equation*}
is known as the \emph{market price of risk}.

Define a new process $\tilde W$ by
\begin{equation*}
  d\tilde W(t) = \theta(t) \, dt + dW(t)\,,
\end{equation*}
and observe
\begin{equation}\label{e:4DS}
  d\paren[\big]{ D(t) S(t) } \, dt = \sigma(t) D(t) S(t) \, d\tilde W(t)\,.
\end{equation}

\begin{proposition}
  If $Z$ is the process defined by
  \begin{equation*}
    Z(t) = \exp\paren[\Big]{
      -\int_0^t \theta(s) \, dW(s) - \frac{1}{2} \int_0^t \theta(s)^2 \, ds
    }\,,
  \end{equation*}
  then the measure $\tilde\P = \tilde\P_T$ defined by $d\tilde\P = Z(T) \, d\P$ is a risk neutral measure.
\end{proposition}
\begin{proof}
  By the Girsanov theorem~\ref{t:4girsanov} we know $\tilde W$ is a Brownian motion under~$\tilde\P$.
  Thus using~\eqref{e:4DS} we immediately see that the discounted stock price is a martingale.
\end{proof}

Our next aim is to develop \emph{risk neutral pricing formula}.
%For instance, $V(T) = (S(T) - K)^+$ represents the payoff of a European call option with strike price $K$.

\begin{theorem}[Risk Neutral Pricing formula]\label{t:4rnp}
  Let $V(T)$ be a $\mathcal F_T$-measurable random variable that represents the payoff of a derivative security, and let $\tilde \P = \tilde\P_T$ be the risk neutral measure above.
  The arbitrage free price at time $t$ of a derivative security with payoff $V(T)$ and maturity $T$ is given by
  \begin{equation}\label{e:4rnp}
    V(t) = \tilde\E \paren[\Big]{\exp\paren[\Big]{-\int_t^T R(s) \, ds} V(T) \given \mathcal F_t }\,.
  \end{equation}
\end{theorem}
\begin{remark}
  It is important to note that the price $V(t)$ above is the actual arbitrage free price of the security, and there is no alternate ``risk neutral world'' which you need to teleport to in order to apply this formula.
  The risk neutral measure is simply a tool that is used in the above formula, which gives the arbitrage free price under the \emph{standard} measure.
\end{remark}

As we will see shortly, the reason for this formula is that under the risk neutral measure, the discounted replicating portfolio becomes a martingale.
To understand why this happens we note
\begin{equation}\label{e:4dS}
  dS(t) = \alpha(t) S(t) \, dt + \sigma(t) S(t) \, dW(t)
    = R(t) S(t) \, dt + \sigma(t) S(t) \, d\tilde W\,.
\end{equation}
Under the standard measure $\P$ this isn't much use, since $\tilde W$ isn't a martingale.
However, under the risk neutral measure, the process $\tilde W$ is a Brownian motion and hence certainly a martingale.
Moreover, $S$ becomes a geometric Brownian motion under $\tilde\P$ with mean return rate of $S$ exactly the same as that of the money market account.
The fact that $S$ and the money market account have exactly the same mean return rate (under~$\tilde\P$) is precisely what makes the replicating portfolio (or any self-financing  portfolio for that matter) a martingale (under $\tilde\P$).

\begin{lemma}
  Let $\Delta$ be any adapted process, and $X(t)$ be the wealth of an investor with that holds $\Delta(t)$ shares of the stock and the rest of his wealth in the money market account.
  If there is no external cash flow (i.e.\ the portfolio is \emph{self financing}), then the discounted portfolio $D(t) X(t)$ is a martingale under~$\tilde\P$.
\end{lemma}
\begin{proof}
  We know 
  \begin{equation*}
    dX(t) = \Delta(t) \, dS(t) + R(t) (X(t) - \Delta(t) S(t)) \, dt\,.
  \end{equation*}
  Using~\eqref{e:4dS} this becomes
  \begin{align*}
    dX(t) &= \Delta R S \, dt + \Delta \sigma S \, d\tilde W
      + R X \, dt - R \Delta S \, dt
    \\
      &= RX \, dt + \Delta \sigma S \, d\tilde W\,.
  \end{align*}
  Thus, by the product rule,
  \begin{align*}
    d(DX) &= D \, dX + X \, dD + d\jqv{D,X}
      = -R D X \, dt + D  RX dt + D \Delta \sigma S \, d\tilde W
    \\
      &= D \Delta \sigma S \, d\tilde W\,.
  \end{align*}
  Since $\tilde W$ is a martingale under $\tilde\P$, $DX$ must be a martingale under $\tilde\P$.
\end{proof}

\begin{proof}[Proof of Theorem~\ref{t:4rnp}]
  Suppose $X(t)$ is the wealth of a replicating portfolio at time~$t$.
  Then by definition we know $V(t) = X(t)$, and by the previous lemma we know $DX$ is a martingale under $\tilde\P$.
  Thus
  \begin{equation*}
    V(t)
      = X(t)
      = \frac{1}{D(t)} D(t) X(t)
      = \frac{1}{D(t)} \tilde\E\paren[\big]{ D(T) X(T) \given \mathcal F_t }
      = \tilde\E\paren[\Big]{
	  \frac{D(T) V(T)}{D(t)} \given \mathcal F_t }\,,
  \end{equation*}
  which is precisely~\eqref{e:4rnp}.
\end{proof}
\begin{remark}
  Our proof assumes that a security with payoff $V(T)$ \emph{has} a replicating portfolio.
  This is true in general because of the \emph{martingale representation theorem}: We know It\^o integrals are martingales. The \emph{martingale representation theorem} is a partial converse which guarantees that any martingale (with respect to the Brownian filtration) can be expressed as an It\^o integral with respect to Brownian motion.

  Now we observe by~\eqref{e:4rnp} that
  \begin{equation*}
    D(t) V(t) = \tilde\E \paren[\Big]{\exp\paren[\Big]{-\int_0^T R(s) \, ds} V(T) \given \mathcal F_t }\,,
  \end{equation*}
  and hence $D(t) V(t)$ is a martingale.
  Thus, the \emph{martingale representation theorem} can be used to express this as an It\^o integral (with respect to~$\tilde W$).
  We can use this to show the existence of a replicating portfolio.
\end{remark}
\begin{remark}
  If $V(T) = f(S(T))$ for some function $f$ and $R$ is not random, then the \emph{Markov Property} guarantees $V(t) = c(t, S(t))$ for some function $c$.
  Equating $c(t, S(t)) = X(t)$, the wealth of a replicating portfolio and using It\^o's formula, we immediately obtain the \emph{Delta hedging rule}
  \begin{equation*}
    \Delta(t) = \partial_x c(t, S(t))\,.
  \end{equation*}
\end{remark}
\section{The Black-Scholes formula}

Recall our first derivation of the Black-Scholes formula only obtained a PDE.
The Black-Scholes formula is the solution to this PDE, which we simply wrote down without motivation.
Use the risk neutral pricing formula can be used to derive the Black-Scholes formula quickly, and independently of our previous derivation.
We carry out his calculation in this section.


Suppose $\sigma$ and $R$ are deterministic constants, and for notational consistency, set $r = R$.
The risk neutral pricing formula says that the price of a European call is
\begin{equation*}
  c(t, S(t)) = \tilde\E\paren[\Big]{
    e^{-r(T-t)}	(S(T) - K)^+ \given \mathcal F_t
  }\,,
\end{equation*}
where $K$ is the strike price.
Since
\begin{equation*}
  S(t) = \exp\paren[\Big]{ \paren[\Big]{ r - \frac{\sigma^2}{2}} t + \sigma \tilde W(t) }\,,
\end{equation*}
and $\tilde W$ is a Brownian motion under $\tilde\P$, we see
\begin{align*}
  c(t, S(t)
    &= e^{-r\tau} \tilde\E\paren[\Big]{\brak[\Big]{
	\exp\paren[\Big]{ \paren[\Big]{ r - \frac{\sigma^2}{2} } T + \sigma \tilde W(T)} -K }^+
      \given \mathcal F_t }
  \\
    &= e^{-r\tau} \tilde\E\paren[\Big]{
      \brak[\Big]{S(t) \exp\paren[\Big]{ \paren[\Big]{r - \frac{\sigma^2}{2}} \tau + \sigma (\tilde W(T) - \tilde W(t))} - K }^+ \given \mathcal F_t }
  \\
    &= \frac{e^{-r \tau}}{\sqrt{2\pi}}
	\int_\R 
	  \brak[\Big]{S(t) \exp\paren[\Big]{ \paren[\Big]{r - \frac{\sigma^2}{2}} \tau + \sigma \sqrt{\tau} y} - K }^+ \, e^{-y^2 / 2} dy\,,
\end{align*}
by the independence lemma.
Here $\tau = T - t$.

Now set $S(t) = x$,
\begin{equation*}
  d_\pm( \tau, x ) \defeq
  \frac{1}{\sigma \sqrt{\tau}}\paren[\Big]{
    \ln\paren[\Big]{\frac{x}{K}}
    + \paren[\Big]{ r \pm \frac{\sigma^2}{2} } \tau
  }\,,
\end{equation*}
and
\begin{equation*}
  N(x)
    = \frac{1}{\sqrt{2\pi}} \int_{-\infty}^x e^{-y^2/2} \, dy
    = \frac{1}{\sqrt{2\pi}} \int_{-x}^\infty e^{-y^2/2} \, dy\,.
\end{equation*}
Observe
\begin{align*}
  c(t, x)
    &= \frac{1}{\sqrt{2\pi}} \int_{-d_-}^\infty 
	x \exp\paren[\Big]{ \frac{-\sigma^2 \tau}{2} + \sigma\sqrt{\tau} y - \frac{y^2}{2} } \, dy
	- e^{-r\tau} K N(d_-)
  \\
    &= \frac{1}{\sqrt{2\pi}} \int_{-d_-}^\infty 
	x \exp\paren[\Big]{ \frac{-(y - \sigma \sqrt{\tau})^2}{2}} \, dy
	- e^{-r\tau} K N(d_-)
  \\
    &= x N(d_+) 
	- e^{-r\tau} K N(d_-)\,,
\end{align*}
which is precisely the Black-Scholes formula.
\end{document}
